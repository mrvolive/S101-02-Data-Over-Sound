\chapter{Results}

\section{Speed}

\subsection{Values}

\begin{itemize}
	\item : 1000 = 0 0 = 1

\end{itemize}

\subsection{LPFilter1's Speed}

\subsubsection{Short File}

\subsubsection{Medium File}

\subsubsection{Long File}

\subsection{LPFilter2's Speed}

\subsubsection{Short File}

\subsubsection{Medium File}

\subsubsection{Long File}

\section{Accuracy}

\subsection{LPFilter1's Accuracy}

\subsubsection{Low n}
% Value sent in cut off frequency is the number of points to average
% Lower n means less filtering

\subsubsection{High n}
% Value sent in cut off frequency is the number of points to average
% Higher n means more filtering

\subsection{LPFilter2's Accuracy}

\subsubsection{Low}
% Lowering the cutoff frequency means averaging from less points
% Lower value means more filtering

\subsubsection{high}
% Upping the cutoff frequency means averaging from more points
% Higher value means less filtering

On the first plot, we have the input that is going into the moving average filter. The input is noisy
and our objective is to reduce the noise. The next figure is the output response of a 3-point
Moving Average filter. It can be deduced from the figure that the 3-point Moving Average filter
has not done much in filtering out the noise. We increase the filter taps to 51-points and we can
see that the noise in the output has reduced a lot, which is depicted in next figure.
Frequency Response of Moving Average Filters of various lengths
We increase the taps further to 101 and 501 and we can observe that even-though the noise is
almost zero, the transitions are blunted out drastically (observe the slope on the either side of the
signal and compare them with the ideal brick wall transition in our input).
Frequency Response:
From the frequency response it can be asserted that the roll-off is very slow and the stop band
attenuation is not good. Given this stop band attenuation,clearly, the moving average filter
cannot separate one band of frequencies from another. As we know that a good performance in
the time domain results in poor performance in the frequency domain, and vice versa. 
In short, the moving average is an exceptionally good smoothing filter (the action in the time
domain), but an exceptionally bad low-pass filter (the action in the frequency domain)